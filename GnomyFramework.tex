\documentclass{article}

%\usepackage[a6paper, margin=5mm]{geometry} % More concise geometry
\usepackage[a4paper, margin=1in]{geometry} % More concise geometry

\usepackage[verbose]{wrapfig}
\usepackage{ragged2e} % For justified text
\usepackage{enumitem} % For custom list item spacing
\usepackage{fontawesome5} % For icons like compass
\usepackage{hyperref} % For hyperlinks
\usepackage[svgnames]{xcolor} % For more color options
\usepackage{titlesec} % For custom section formatting
\usepackage{tikz} % For drawing (if you include illustrations later)
\usepackage[most]{tcolorbox} % For beautiful boxes
\usepackage{microtype} % For better typography (font protrusion, character tracking etc.)
\usepackage{graphicx} % For including images

% Define Victorian-inspired colors
\definecolor{VictorianDeepGreen}{HTML}{2F4F4F} % Dark Slate Gray
\definecolor{VictorianPlum}{HTML}{673147} % Muted Plum/Deep Rose
\definecolor{VictorianTeal}{HTML}{008080} % Teal
\definecolor{VictorianGold}{HTML}{B8860B} % Dark Goldenrod
\definecolor{VictorianBrown}{HTML}{8B4513} % Saddle Brown
\definecolor{VictorianGrey}{HTML}{4F4F4F} % Dark Grey
\definecolor{VictorianCream}{HTML}{F5F5DC} % Beige

% Creative Commons License (CC BY 4.0)
% Using tcolorbox for a visually appealing license notice
\tcbset{
  licensebox/.style={
    colback=VictorianCream, % Creamy background
    colframe=VictorianGrey, % Dark Grey frame
    boxsep=5pt, % Space between content and box
    arc=4pt, % Rounded corners
    boxrule=0.5pt, % Thin border
    left=10pt,right=10pt,top=5pt,bottom=5pt, % Padding
    fonttitle=\bfseries,
    fontupper=\small,
    halign title=center,
    valign=middle,
    nobeforeafterifvbox
  }
}

\newcommand{\printgnomylicense}{%
\begin{tcolorbox}[licensebox]
  \centering
  \textbf{Creative Commons Attribution 4.0 International License (CC BY 4.0)} \\
  This work, "The Gnomy Framework version 1.1: Cultivating Emotional \& Strategic Clarity," by Rickard V. Hultgren, is licensed under a Creative Commons Attribution 4.0 International License. To view a copy of this license, visit \url{http://creativecommons.org/licenses/by/4.0/}. \\
  \faIcon{creative-commons} \faIcon{creative-commons-by}
\end{tcolorbox}
}

% Custom section formatting with more visual flair using Victorian colors
\titleformat{\section}[block]{\centering\color{VictorianDeepGreen}\huge\bfseries}{\thesection.}{1em}{\color{VictorianTeal}\rule{\linewidth}{0.8pt}\\[1ex]}
\titlespacing*{\section}{0pt}{3.5ex plus 1ex minus .2ex}{2.3ex plus .2ex}

\titleformat{\subsection}[hang]{\color{VictorianPlum}\Large\bfseries}{\thesubsection.}{1em}{}
\titlespacing*{\subsection}{0pt}{3.25ex plus 1ex minus .2ex}{1.5ex plus .2ex}

\begin{document}

\begin{center}
    \vspace*{0em} % Add some vertical space
    \color{VictorianBrown}\textbf{\Huge The Gnomy Framework \version{1.2}} \\
    \vspace{0.5em} % Space between title and author
    \color{VictorianGrey}\large By Rickard V. Hultgren: Cultivating Emotional \& Strategic Clarity
    \vspace{0em} % Space after author line
\end{center}

\printgnomylicense % Print the custom license box

\vspace{1em} % Space after the license box

\justify % Ensures the text is justified
The Gnomy Framework is a structured yet flexible tool designed to enhance emotional self-leadership, coaching, reflection, and narrative transformation. It is particularly powerful when navigating complexity, change, or interpersonal dynamics. Here is a clear breakdown of its purpose and applications, categorized by intent and user contexts.

---

\section{Core Purpose of the Gnomy Framework}
The Gnomy Framework provides a structured journey for individuals and groups to:
\begin{itemize}[noitemsep,topsep=0pt]
    \item \textbf{Anchor} themselves before reacting or making decisions.
    \item \textbf{Unravel} internal narratives and external influences.
    \item \textbf{Envision} future possibilities with emotional intelligence.
    \item \textbf{Strategize} risks, motivations, and responsibilities.
    \item \textbf{Iterate} and evolve.
\end{itemize}
It supports a holistic process of emotional insight, cognitive clarity, and personal strategy-making. It's an ideal tool for coaching, personal growth, mental health support, group reflection, and leadership training.

---

\section{Summary of Gnomy Component Roles}
\begin{tabular}{|p{0.2\textwidth}|p{0.35\textwidth}|p{0.35\textwidth}|}
    \hline
    \textbf{Stage} & \textbf{Purpose} & \textbf{Function} \\
    \hline
    \textbf{G — Grounding} & Establish presence and perspective. & Breath, pause, curiosity, perspective framing using \textbf{FIND the LENS}. \\
    \textbf{N — Narrative} & Understand internal drives and tensions. & Identify pain points and feelings towards the challenge using the \textbf{LUX Lantern model}. \\
    \textbf{O — Others} & Map relational influences and shared identity. & Understand self and others across time using \textbf{CROWS}. \\
    \textbf{M — More} & Identify inner and outer resources. & Leverage strengths, relationships, knowledge using \textbf{SPARK}. \\
    \textbf{Y — Yikes!} & Strategize through risks and change. & Assess risk spectrum, derive action, prioritize, evolve using \textbf{SAGE}. \\
    \hline
\end{tabular}

\section{The Gnomish Journey: A Metaphorical Compass}


\vspace{1em} % Add some vertical space below the section title

\noindent
\begin{minipage}[t]{0.3\linewidth} % Adjust width as needed for the image
    \includegraphics[width=\linewidth]{GNOMY.png}
\end{minipage}\hspace{0.5cm}\begin{minipage}[t]{0.65\linewidth} \vspace{-0.6\linewidth}% Adjust width as needed for the text, leaving some space between
    \justifying % Justify the text within this minipage
    Imagine this: A Gnomish Journey of Pond-Bound Reflection. A small gnome stands on a delicate, hybrid plant — perhaps a blend of strawberry and water lily — nestled within a tranquil pond. This pond embodies a specific context, while the individual plant represents a distinct situation. Plants in this pond are interconnected by slender tendrils. The leaves and the tendrils of the platns havew sharp thorns (pain points of the situation). This metaphor serves as a cognitive anchor, associating specific actions or mental states with distinct metaphorical elements, thereby becoming a quick mental shortcut for recalling and applying the framework's principles, especially during moments of stress or complexity.
\end{minipage}

\vspace{1em}
\subsection*{Understanding the Symbol System}
The visual metaphor of the Gnomy Framework operates through an integrated system of symbols, where each element carries a specific meaning and interacts dynamically with the others:

\begin{itemize}[noitemsep,topsep=0pt]

    \item \textbf{The Plant (Situation):}  
    Each plant represents a distinct situation. \textit{Tendrils} stretch between plants, symbolizing the actions or transitions required to move from one situation to another. \textit{Thorns} mark points of friction, frustration, or pain—places where the gnome's competence or autonomy feels blocked. As the gnome develops workarounds through emotional adaptation or cognitive reappraisal, new, lighter tendrils grow around or over the thorns, illustrating growth and learning.  
    Some tendrils connect to other plants, while others loop back to different parts of the same plant, representing reflection or internal problem-solving. Occasionally, a tendril remains loose until the gnome throws it outward, seeking something for it to attach to. Certain gnomes are naturally resistant or immune to specific thorns.

    \item \textbf{The Lantern Glow (Emotion):}  
    The lantern’s light changes in color and radius depending on the gnome’s emotional tone and clarity. When the light is dim or narrow, the gnome perceives only nearby thorns and becomes entangled—representing emotional tunnel vision. When the light broadens or becomes more colorful, the gnome perceives more tendrils, symbolizing emotional understanding and access to alternative perspectives. The glow also reveals previously hidden tendrils.

    \item \textbf{The Crows (Relatedness):}  
    Crows carry messages written in glowing powder, facilitating emotional and social exchange between gnomes. When a crow delivers this powder to another gnome’s lantern, it illuminates specific parts of the plant—showing how relationships influence attention, meaning, and emotional tone. The powder shimmers like pollen dust, revealing hidden tendrils or even transforming thorns into buds when empathy or understanding is reached.

    \item \textbf{The Tendrils (Perceived Possibilities):}  
    Tendrils may be latent (already present but unseen) or projected (actively lassoed outward by the gnome). When the gnome’s lantern brightens or receives a crow’s message, hidden tendrils become visible—representing insight or new connections. When the gnome throws a tendril outward, it signifies creative problem-solving or hope—the act of imagining a possible bridge before it becomes real. Successfully attached tendrils pulse and thicken like living cables.

    \item \textbf{The Fallen Leaves (Action Pathways):}  
    Each gnome maintains a heap of fallen leaves, representing collected skills or competences. These leaves can serve as small ferries across the pond water, with tendrils acting as their guiding cables—allowing the gnome to move toward another, or sometimes the same, plant. The leaf pathways embody the gnome’s practical ability to act upon insight.

    \item \textbf{The Spark (Agency):}  
    The spark represents the act of doing—the moment of agency and initiative. Hovering above the gnome’s left hand, the glowing spark stirs a gentle wind that gathers dried leaves from the leaf heap. This symbolizes the uncovering of inner resources and the use of learned competences to navigate the garden. The gathered leaves become vessels for movement: some leading outward to new plants (external change), others curling inward (internal reflection).

\end{itemize}


\textbf{Dynamic Interactions:} The messages carried by crows change both the plant and lantern glow, as well as how clearly the gnome sees the tendrils to other plants. When a crow arrives with glowing powder, the lantern changes hue; this glow spreads along tendrils, softening some thorns and revealing new tendrils. The gnome then notices that some leaves (decisions) can now connect to other plants—paths that were invisible before.

\textbf{Frustration and Growth:} The framework visualizes learning as a transformation process: thorns represent blocked needs, covering thorns with new tendrils shows coping or adaptation, and thorns transforming into buds represents genuine emotional growth. When multiple gnomes exchange messages, their collective pattern can reshape the plant: tangled roots straighten, or new shared branches grow—showing social resonance and narrative alignment.

\subsection*{Key Metaphorical Elements:}
\begin{itemize}[noitemsep,topsep=0pt]
    \item \textbf{FIND the LENS:} The Gnome wears glasses. The right lens helps see the \textit{antagonist} in the situation (who this is about), while the left lens helps see the \textit{causing force} (deuteragonist or antagonist). These glasses represent gaining perspective and discerning different appraisals. As the gnome adjusts their glasses, the lantern's glow shifts, revealing previously hidden tendrils or softening visible thorns.
    \item \textbf{LUX Lantern model:} In their left hand, the gnome holds a lantern of current feeling. From bottles in their belt, filled with luminescent powder of different colours, the gnome pours some powder into the lantern, and it begins to glow. This symbolizes illuminating your inner narrative and emotional landscape. The lantern's light changes color and radius—when dim or narrow, only nearby thorns are visible; when broad and colorful, multiple tendrils become apparent, expanding perceived possibilities.
    \item \textbf{CROWS:} On the right shoulder, a black crow perches confidently, ready to fly to other gnomes with messages. This embodies understanding how others shape your story and considering external perspectives. When a crow delivers a message carrying glowing powder, it temporarily alters the receiver's lantern color, revealing new aspects of their plant and making hidden tendrils visible—showing how relationships transform perception and emotional awareness.
    \item \textbf{SPARK:} Above the left hand, the glowing spark is floating in the air, manipulated by the gnome's twisting hand, creating a gentle wind that gathers dried leaves from the gnome's leaf-heap. This symbolizes uncovering resources and possibilities, using them to navigate. These leaves become the vessels for movement—some leading outward to new plants (external change), others curling inward (internal reflection).
    \item \textbf{SAGE:} In their right hand, the gnome holds a single sage leaf, dry and bitter on the tongue. It's a reminder: wisdom isn't always sweet. It comes with limits, trade-offs, and hard-earned truths. This represents facing risks and choosing wisely. The sage grows along the tendrils between plants, marking the path of growth and the places where thorns have been transformed into learning.
\end{itemize}
And so, this gnome walks—leaving footprints shaped by emotion, relationship, strength, and restraint. Every step is a journey through GNOMY: Grounding, Narrative, Others, More, and Yikes! As the gnome moves, thorns may soften or transform into buds through emotional growth, new tendrils may sprout from collective exchanges, and the ever-shifting lantern glow reveals pathways that were always there, waiting to be seen.

---

\section{The Gnomy Framework: Your Step-By-Step Guide}
The Gnomy Framework is designed to be a practical tool. This section provides detailed, actionable prompts for each step, transforming theoretical concepts into tangible user experiences.

\subsection{Step 0: Grounding (FIND the LENS) — Find Your Inner Position}
\textbf{FIND the LENS} helps you regain perspective by pausing to explore goals, emotions, and responsibility with intention. Before diving into analysis or action, ground yourself in the present moment to regain perspective. Grounding prepares your mind and body for reflective, intentional thinking. Use the \textbf{FIND} sequence to anchor yourself and then apply the \textbf{LENS} to discern different appraisals.

\begin{center}
% Placeholder for FIND the LENS diagram
% This diagram could visually represent the gnome wearing glasses, with prompts for each FIND step around them.
\end{center}

\begin{itemize}[noitemsep,topsep=0pt]
    \item \textbf{F — Feel your breath/urge}: Embodied grounding. Helps interrupt reactivity and build emotional awareness before analysis.
    \begin{itemize}[noitemsep,topsep=0pt]
        \item \textit{Now feel your urge—and others' urges.}
        \item \textbf{$\blacktriangleright$ Appraisal dimension: Goal Relevance}
        \item \textbf{$\blacktriangleright$ Ask:}
        \begin{itemize}[noitemsep,topsep=0pt]
            \item "What urge is arising in me right now? What is the physical sensation associated with it?"
            \item "What is the emotional signal behind it? (e.g., anger, fear, sadness, excitement)"
            \item "What urge might be arising in the other people involved? How might their physical or emotional state be similar or different from mine?"
        \end{itemize}
        \item \textbf{$\blacktriangleright$ Insight:} Emotions signal that a \textbf{goal or value is at stake}—yours or someone else's. Tune into that signal. This aligns with DBT's emphasis on emotional awareness \cite{linehan1993cognitive}.
    \end{itemize}

    \item \textbf{I — Introduce pause}: Goal-mapping and perspective expansion. Surfaces implicit motivations and helps distinguish between short-term urges and deeper goals.
    \begin{itemize}[noitemsep,topsep=0pt]
        \item \textit{Silently say: "It's just in my mind." This reminds you that goals live inside individual minds.}
        \item \textit{Now ask: "What else can I find?"}
        \item \textbf{$\blacktriangleright$ Appraisal dimensions: Goal Congruence \& Hidden Goals}
        \item \textbf{$\blacktriangleright$ Ask:}
        \begin{itemize}[noitemsep,topsep=0pt]
            \item "What other goals (both explicit and implicit) are connected to the topic for me?"
            \item "What are the loudest or most obvious goals right now, and what are the quieter, underlying ones?"
            \item "What goals am I affected by—but might not be fully aware of—from others involved or from the broader context?"
        \end{itemize}
        \item \textbf{$\blacktriangleright$ Insight:} We often focus on the loudest or most conscious goals, but \textbf{conflicting or background goals may be shaping your reactions}—and others'. This step expands on Lazarus's primary appraisal of goal congruence \cite{lazarus1984stress}.
    \end{itemize}

    \item \textbf{N — Notice urgency}: Action-orientation + systems thinking. Asks what can be done constructively, not just what is felt. Encourages agency over rumination.
    \begin{itemize}[noitemsep,topsep=0pt]
        \item \textit{Ask yourself: What can I do about the expressed and non-received goals?}
        \item \textbf{$\blacktriangleright$ Appraisal dimension: Coping Potential}
        \item \textbf{$\blacktriangleright$ Ask:}
        \begin{itemize}[noitemsep,topsep=0pt]
            \item "Which goals feel most urgent or pressing for me and for others?"
            \item "What specific actions are possible right now to address these urgent goals, even small ones?"
            \item "What's outside of my immediate control or influence in this situation? How can I accept that?"
        \end{itemize}
        \item \textbf{$\blacktriangleright$ Insight:} Urgency often comes from feeling like we can't act on a critical goal. \textbf{Noticing what's doable helps restore emotional agency.} This reflects Lazarus's secondary appraisal of coping potential \cite{lazarus1984stress}.
    \end{itemize}

    \item \textbf{D — Discern Stakes (and Divine Compassion)}: Narrative humility, shared responsibility. Avoids blame, embraces complexity, promotes collective insight and equity thinking. Instead of assigning responsibility, \textbf{Discern Stakes} invites a humble inquiry into who is experiencing the highest emotional or physical vulnerability in this situation. From a spiritual appraisal perspective, this isn't about blame, but recognizing the deepest "sacred concerns" or "spiritual goals" at risk.
    \begin{itemize}[noitemsep,topsep=0pt]
        \item \textbf{$\blacktriangleright$ Appraisal dimensions: Goal Significance \& Existential Threat.}
        \item \textbf{$\blacktriangleright$ Ask:}
        \begin{itemize}[noitemsep,topsep=0pt]
            \item "For whom are the \textbf{emotional stakes highest} right now? (e.g., fear of loss, profound grief, intense anxiety). What specific emotions indicate these high stakes?"
            \item "Who faces the \textbf{greatest physical vulnerability} or potential harm in this situation?"
            \item "What \textbf{core values or sense of purpose} feel most threatened for different individuals involved? (This links emotions to our relationship with the sacred or ultimate concerns, extending traditional appraisal theory)."
            \item "How can I approach this discernment with \textbf{deep compassion and humility} for all experiencing high stakes, even those whose actions I might disagree with?"
        \end{itemize}
        \item \textbf{$\blacktriangleright$ Insight:} Emotions are signals that something deeply valuable, perhaps even sacred, is at stake. By humbly discerning these vulnerabilities across all involved, we cultivate empathy and open a path towards responses rooted in compassion rather than judgment, aligning with a spiritual understanding of interconnected suffering and well-being.
    \end{itemize}
\end{itemize}

---

\subsubsection*{\textbf{Applying the LENS: Layered Emotive Narratives of Stakes}}
\textbf{LENS} This framework emphasizes understanding a situation by viewing it through different "lenses." This component represents a unique extension of Lazarus's appraisal theory, introducing relational and temporal dimensions to emotional appraisal.

First, establish the timeframe of the situation: past, current, or future.

Next, identify the protagonist as the individual or group with the most to gain or save in the narrative. The deuteragonist or antagonist (depending on whether they intend to help or harm the protagonist's cause) is then identified as the party working towards or against the protagonist's goal.

Each letter in this framework represents a pair of contrasting appraisals—one from the protagonist's perspective and the other from the antagonist's—at different points in time. These appraisals draw upon Lazarus's concepts of primary and secondary appraisals, but are uniquely framed within Gnomy's relational context.

\begin{itemize}[noitemsep,topsep=0pt]
    \item \textbf{L - Loss (Past) vs. Legitimacy (Past)}
    \begin{itemize}[noitemsep,topsep=0pt]
        \item \textbf{Protagonist (Your View):} Focus on the \textbf{Loss} (Harm/Loss) endured in the past. \textit{Prompt: "What specific harm or loss have I experienced or perceived in the past related to this situation?"}
        \item \textbf{Antagonist (Their View):} Focus on the \textbf{Legitimacy} of their past actions or the perceived "credit" they deserve for outcomes you see as harmful. \textit{Prompt: "From their perspective, what past actions do they believe were justified or what credit do they feel they deserve for past outcomes, even if I see them as harmful?"}
        \item \textit{Paired Appraisals:} Harm/Loss (Primary) vs. Blame/Credit (Secondary, from their view).
    \end{itemize}

    \item \textbf{E - Effort (Present) vs. Entitlement (Present)}
    \begin{itemize}[noitemsep,topsep=0pt]
        \item \textbf{Protagonist (Your View):} Focus on the \textbf{Effort} (Challenge) being exerted in the present to overcome difficulties. \textit{Prompt: "What challenges am I currently facing, and what effort am I putting in to overcome them?"}
        \item \textbf{Antagonist (Their View):} Focus on their \textbf{Entitlement} or belief in their \textit{current} rightness, or the perception that your challenge is due to their justified actions. \textit{Prompt: "From their perspective, what are they doing now that they believe is justified, and why might they feel entitled to their current position or actions?"}
        \item \textit{Paired Appraisals:} Challenge (Primary) vs. Coping Potential (Secondary, their perceived control/justification).
    \end{itemize}

    \item \textbf{N - Nuisance (Future) vs. Necessity (Future)}
    \begin{itemize}[noitemsep,topsep=0pt]
        \item \textbf{Protagonist (Your View):} Views their actions/existence as a \textbf{Nuisance} (Threat) for the future. \textit{Prompt: "How do I perceive their potential future actions or presence as a threat or nuisance to my goals or well-being?"}
        \item \textbf{Antagonist (Their View):} Views their future actions as a \textbf{Necessity} for their goals, even if it threatens yours. \textit{Prompt: "From their perspective, what future actions do they see as absolutely necessary for their own goals, even if it impacts me negatively?"}
        \item \textit{Paired Appraisals:} Threat (Primary) vs. Future Expectations (Secondary, their predicted justified outcomes).
    \end{itemize}

    \item \textbf{S - Safety (Your Goal) vs. Supremacy (Their Goal)}
    \begin{itemize}[noitemsep,topsep=0pt]
        \item \textbf{Protagonist (Your View):} Primarily seeking \textbf{Safety} and well-being. \textit{Prompt: "What is my ultimate goal in this situation — what does safety or well-being look like for me?"}
        \item \textbf{Antagonist (Their View):} Primarily seeking \textbf{Supremacy} or dominance in the situation. \textit{Prompt: "What does their ultimate goal appear to be — is it about control, winning, or asserting dominance?"}
        \item \textit{General Goal Contrast: This is more of an overarching contextual goal than a specific appraisal, but it helps frame the differing drivers.}
    \end{itemize}
\end{itemize}

---

\subsection{Step 1: Narrative of Emotion and Meaning}

\subsubsection{1.1 Narrative (LUX) — Exploring Emotions and Their Roots}
Emotions are signals, not noise. They reveal unmet needs, broken expectations, and meaningful desires.  
The \textbf{LUX Lantern Model} supports the exploration of emotional narratives by illuminating one’s inner landscape along three key axes.  
This model uniquely integrates neurochemical correlates with the Kano Model’s feature categories to help interpret emotional responses.

Emotions arise both from one’s own situation and from the narratives shared by others. When another gnome sends a message, it is written with \textit{glowing powder}—emotional content that enters your lantern and influences how you perceive your surroundings.  
If you extend a \textit{tendril} toward another gnome’s plant, you can travel along that tendril, drawing closer, allowing your lantern to partially illuminate the situation that the other gnome is experiencing.

\begin{center}
% Placeholder for LUX Lantern diagram
% (Diagram: a lantern divided into three colored sections labeled L, U, and X,
% each linked to a neurochemical and a Kano feature category)
\end{center}

\begin{itemize}[noitemsep,topsep=0pt]

    \item \textbf{L — Longing (Dopamine $\times$ Kano’s Attractive Features)}  
    This axis represents desire, hope, and intrinsic motivation. What do you long for in this situation? What possibilities pull you forward or spark joy?  
    \textit{Proposed link:} Dopamine is associated with reward-seeking and motivation, aligning with Kano’s “Attractive Features”—aspects that create delight when present but are not expected.  
    \begin{itemize}[noitemsep,topsep=0pt]
        \item \textbf{$\blacktriangleright$ Ask:} “What do I genuinely appreciate in this situation?”  
        \item \textbf{$\blacktriangleright$ Ask:} “What do I hope or wish for from this situation?”  
    \end{itemize}

    \item \textbf{U — Usefulness (Serotonin $\times$ Kano’s Performance Features)}  
    This axis concerns reliability, control, and competence. How does this situation affect your sense of self-worth, stability, or capability? What feels steady and supportive?  
    \textit{Proposed link:} Serotonin is linked to well-being, stability, and confidence—qualities that parallel Kano’s “Performance Features,” which generate satisfaction when functioning properly.  
    \begin{itemize}[noitemsep,topsep=0pt]
        \item \textbf{$\blacktriangleright$ Ask:} “Which ‘performance features’—the expected functions of this situation—are working or not working, and how does that affect me?”  
        \item \textbf{$\blacktriangleright$ Ask:} “In what ways do I manage these challenges? (Which tendrils help me avoid being hurt by the thorns?)”  
    \end{itemize}

    \item \textbf{X — X-Friction (Noradrenaline $\times$ Kano’s Must-Be/Reverse Features)}  
    This axis addresses stress, threat, and unmet basic needs. What obstacles, irritations, or pressures are you facing? What feels essential—something whose absence causes strong distress?  
    \textit{Proposed link:} Noradrenaline is associated with vigilance and stress response, corresponding to Kano’s “Must-Be” or “Reverse Features,” which, when absent, produce dissatisfaction or alarm.  
    \begin{itemize}[noitemsep,topsep=0pt]
        \item \textbf{$\blacktriangleright$ Ask:} “What specific threats, irritations, or pressures am I experiencing right now?”  
        \item \textbf{$\blacktriangleright$ Ask:} “What challenges do I want to overcome? (Which thorns on the current plant am I confronting?)”  
    \end{itemize}

\end{itemize}



\subsubsection{1.2 Others (CROWS) — Who Else Is Part of This Emotional Story?}
We are shaped by relationships—past, present, and imagined. A crow will fly to other gnomes, inviting them to transfer their insights to your understanding. Ask \textbf{CROWS} to place your experience in a broader social context, fostering mentalization and perspective-taking.

Crows know from what situation/position they fly. They try to find a similar lantern light on the same plant. Each message a crow carries contains glowing powder that temporarily changes the receiver's lantern color, revealing or highlighting certain parts of their plant—showing how relationships direct attention and emotional tone. When the powder shimmers across the plant surface, it can make specific leaves and tendrils visible, or even transform thorns into buds when empathy is reached, demonstrating the transformative power of relational understanding.

\begin{itemize}[noitemsep,topsep=0pt]
    \item \textbf{C — Current others}: Who else is involved now? How do their perspectives or emotions interact with yours? \textit{Prompt: "Who are the key people currently involved in this situation? What might they be thinking, feeling, or needing right now? How might their messages (metaphorically carried by crows) change the color of my lantern and reveal new aspects of this plant?"}
    \item \textbf{R — Retrospective self}: How would your past self interpret or feel about this? What lessons from past experiences are relevant now? \textit{Prompt: "If my past self (e.g., 5 years ago, or during a similar challenge) looked at this situation, what would they think or feel? What lessons did I learn then that apply now? What 'glow powder' from past experiences can illuminate current hidden tendrils?"}
    \item \textbf{O — Optional future self}: What emotional legacy are you creating? What might your future self think or feel? \textit{Prompt: "Imagine your future self, perhaps a year from now, looking back at this moment. What emotional legacy do you want to have created? What advice would your wise future self give you? Which thorns might they see as transformed into buds?"}
    \item \textbf{W — Wished-for others}: Are there people you hope to connect with or impress? Fantasies, role models, or longings? \textit{Prompt: "Are there people (real or imagined, like role models or mentors) whose opinions or approval I value in this context? What kind of connection or impact do I wish to have with them? What new tendrils might their perspective reveal?"}
    \item \textbf{S — Shared past}: What shared experiences or histories are influencing this moment? \textit{Prompt: "What shared experiences, histories, or agreements (or disagreements) are influencing this moment for all involved? How does our collective history shape current dynamics and emotions? When multiple gnomes exchange messages, how might their collective pattern reshape the plant—straightening tangled roots or growing new shared branches?"}
\end{itemize}

\subsection{Step 2: Reframe Through Shared Competence and Action}

\subsubsection{2.1 More (SPARK) — Uncover Resources, Strengths, and Options}
Imagine the gnome's current situation as a plant in a pond. To move to another "situation" (represented by a different plant), the gnome needs a leaf boat—an old, dried leaf from their personal leaf heap. The gnome steers a floating spark in the air by twisting and gesturing with their left hand. This spark, when moved circularly, heats the surrounding air, creating a gentle wind. With this wind, the gnome directs a dried leaf from their heap to the water's edge of their current plant. Stepping onto this leaf, the gnome then uses it as a cable ferry, pulling along a tendril stretched between the plants to cross to the next. This leaf boat is the asset that enables the gnome to navigate from one situation to another.

These leaves represent action pathways: some lead outward to new plants (external change and transitions), while others fold back onto the same plant (internal reflection or reappraisal loops). The tendrils between plants can be latent—already present but unseen—or actively projected when the gnome throws them out, representing creative problem-solving and hope. When the gnome's lantern brightens through emotional awareness or receives glow powder from a crow's message, hidden tendrils appear, revealing insights and connection discoveries. Successfully connected tendrils pulse and thicken like living cables, showing strengthened pathways to resources and possibilities.

This stage leverages principles from solution-focused brief therapy and positive psychology. From this awareness, you can move towards possibility. Expand your narrative understanding through the wisdom of other "gnomes" (people or external knowledge) or your own inner resources. Use SPARK to inventory the personal and interpersonal resources available to you:

\begin{itemize}[noitemsep,topsep=0pt]
    \item \textbf{S — Strengths}: What abilities, traits, or skills do you bring to this moment? \textit{Prompt: "What are my core strengths, talents, or positive qualities that can help me navigate this situation? (e.g., resilience, creativity, empathy, analytical thinking). Which of these strengths might help me see new tendrils or transform thorns into opportunities?"}
    \item \textbf{P — Possibilities}: What courses of action exist? Be creative—don't filter options yet. \textit{Prompt: "Brainstorm at least 3-5 different courses of action, even if they seem unconventional or difficult. What are all the potential ways forward? Which leaves from my heap might serve as vessels to other plants, and which might curl inward for deeper reflection?"}
    \item \textbf{A — Assets}: What tangible and intangible resources (e.g., tools, time, information, shared community resources) can you rely on? \textit{Prompt: "What tangible assets (e.g., time, money, specific tools, information) or intangible assets (e.g., reputation, trust, past successes) do I have access to? What dried leaves in my heap are ready to be used?"}
    \item \textbf{R — Resourced Relationships}: Who can walk with you in this? Who offers strength, support, challenge, or insight? \textit{Prompt: "Who are the people in my network (friends, family, colleagues, mentors) who can offer support, advice, or a different perspective? Who can I reach out to? Which crows might carry messages that illuminate new tendrils or soften existing thorns?"}
    \item \textbf{K — Knowledge \& Experience}: What lessons have you already learned? What frameworks or facts do you already hold? \textit{Prompt: "What relevant knowledge or past experiences do I possess that could inform my approach? What frameworks or facts do I already hold that are relevant to this challenge? What 'glow powder' from past learning can I apply to brighten my current lantern?"}
\end{itemize}

\subsubsection{2.2 Yikes! (SAGE) — Face Risks, Limits, and Priorities}
On the tendril there are sage leaves growing, symbolizing the wisdom and often bitter truths involved in facing risks. These sage leaves mark the path between plants, reminding the gnome that growth requires navigating thorns—points of friction, frustration, or pain where competence or autonomy feels blocked. As the gnome develops workarounds through emotional adaptation or cognitive reappraisal, new lighter tendrils grow around or over these thorns. Sometimes, through the process of wisdom-seeking and emotional growth, thorns transform into buds—representing learning, adaptation, and new possibilities emerging from past pain.

Before committing to action, reflect on risk, responsibility, and refinement. Use \textbf{SAGE} to evaluate outcomes and adapt. This stage integrates strategic foresight, stress inoculation training, and ACT principles.

\begin{itemize}[noitemsep,topsep=0pt]
    \item \textbf{S — Spectrum}: Define three kinds of outcomes:
    \begin{itemize}[noitemsep,topsep=0pt]
        \item \textbf{Scary}: What's the worst-case scenario? What would feel like a failure? \textit{Prompt: "If I take action, what is the absolute worst outcome I can imagine? What would be the biggest failure or loss? What thorns loom largest in my dimmed lantern light?"}
        \item \textbf{Sufficient}: What's "good enough" to move forward? \textit{Prompt: "What would be a 'good enough' outcome? What's the minimum acceptable result that would allow me to move forward? At what point do some thorns become navigable rather than blocking?"}
        \item \textbf{Successful}: What would a meaningful win look like? \textit{Prompt: "What would a truly meaningful win or ideal outcome look like for me in this situation? How might thorns transform into buds, and what new tendrils might become visible?"}
    \end{itemize}
    \item \textbf{A — Actions}: For each outcome, what are possible reactions or strategies?
    \begin{itemize}[noitemsep,topsep=0pt]
        \item \textbf{Avoidance}: What might I try to escape or delay? \textit{Prompt: "What tendencies do I have to avoid or delay action in this situation, and what are the consequences of that avoidance? Am I staying too close to familiar thorns rather than exploring new tendrils?"}
        \item \textbf{Anchoring}: What stable actions keep me grounded under stress? \textit{Prompt: "What stable, consistent actions can I take to stay grounded and resilient, even if the situation becomes stressful? How can I maintain my lantern's steady glow even when thorns are present?"}
        \item \textbf{Aspiration}: What bold or creative steps move me toward success? \textit{Prompt: "What bold, creative, or values-aligned steps can I take to move towards my 'Successful' outcome? What new tendrils am I willing to throw out, even before they're proven real?"}
    \end{itemize}
    \item \textbf{G — Gauge}: Assess urgency and importance. \textit{Prompt: "Based on my desired outcomes and potential actions, which actions are most urgent and most important right now? What's the immediate next step? Which leaf should I step onto first?"}
    \item \textbf{E — Evolution}: The world changes—and so do you. This step mirrors the OODA loop for continuous adaptation.
    \begin{itemize}[noitemsep,topsep=0pt]
        \item \textbf{Examine}: What feedback are you receiving? \textit{Prompt: "What new information, reactions, or feedback am I observing from my actions or the environment? How has my plant changed? What new messages have crows brought?"}
        \item \textbf{Evaluate}: What's working, what's not? \textit{Prompt: "Based on this feedback, what aspects of my plan or actions are working well, and what is not working as expected? Have new tendrils strengthened or have some thorns become more prominent?"}
        \item \textbf{Edit}: What needs adjusting or dropping? \textit{Prompt: "What specific adjustments do I need to make to my approach, or what elements should I drop entirely? Should I change which leaf I'm using or adjust my lantern to see different tendrils?"}
        \item \textbf{Expand}: Connect to the pain-points. Reflect on what changed through your Gnomy process. What pain points were resolved? What challenges remain or new tensions arose? Who benefited, who was affected, and what became clearer? Share this with those impacted by the results. This step isn't just closure—it's a bridge to the next cycle, grounding future action in real insight and shared learning. \textit{Prompt: "Reflect on the entire Gnomy process: What thorns transformed into buds? What pain points were resolved or eased? What new insights did I gain? What challenges or tensions remain, or have new ones arisen? Who benefited from my actions, and who was affected? What became clearer for me and for others? How will I share these insights with those impacted? When multiple gnomes exchanged messages, how did their collective pattern reshape plants—straightening tangled roots or growing new shared branches?"}
    \end{itemize}
\end{itemize}

\subsection{Feedback Loop}
If your plan feels shaky, your emotions shift, or new obstacles appear, it's like the gnome realizing their current plant isn't stable or the path forward is unclear. In such moments, the gnome doesn't despair; instead, they might return to \textbf{Step 0 (Grounding)} to steady their footing on their current plant, or revisit \textbf{Step 1 (Narrative)} to re-evaluate the story of where they are and where they want to go. 

The Gnomy Framework, much like the interconnected plants in the pond, is cyclical and flexible—designed for ongoing clarity, growth, and recalibration as you navigate from one "situation-plant" to the next. When the gnome's lantern dims or thorns seem insurmountable, returning to earlier steps can brighten the light, reveal hidden tendrils, or allow new messages from crows to transform perspective. Sometimes leaves that once seemed to lead forward need to curl inward for deeper reflection before true progress can be made. The framework honors this natural rhythm of exploration, retreat, and renewed advance—recognizing that thorns don't disappear instantly but can be gradually transformed into buds through repeated engagement and evolving understanding.

---

\section{What Gnomy Should Be Used For}

\subsection{Coaching \& Mentorship}
To help clients:
\begin{itemize}[noitemsep,topsep=0pt]
    \item Gain clarity in emotionally charged or complex situations.
    \item Shift from rumination to structured insight.
    \item Reframe failures, risks, or uncertainties into growth paths.
    \item Clarify responsibilities and action options in relational contexts.
\end{itemize}

\subsection{Self-Reflection \& Emotional Navigation}
For individuals to:
\begin{itemize}[noitemsep,topsep=0pt]
    \item Understand how personal stories and social influences shape decisions.
    \item Ground themselves during stress or at key decision points.
    \item Identify emotional needs, risks, and motivations.
    \item Develop resilience through narrative self-work and emotional agency.
\end{itemize}

\subsection{Team \& Group Dynamics}
For use in teams to:
\begin{itemize}[noitemsep,topsep=0pt]
    \item Reflect on past collaboration patterns and improve communication.
    \item Clarify individual and shared goals, risks, and unspoken assumptions.
    \item Create psychological safety by mapping different emotional truths.
    \item Balance personal contributions with team-wide needs.
\end{itemize}

\subsection{Leadership \& Strategy Design}
To support leaders in:
\begin{itemize}[noitemsep,topsep=0pt]
    \item Balancing rational planning with underlying emotional currents.
    \item Anticipating team member perspectives and motivations.
    \item Making decisions aligned with both vision (aspiration) and risk awareness.
    \item Creating roadmaps that account for psychological barriers and enablers.
\end{itemize}

---

\section{What Gnomy Could Be Used For (Extensions and Creative Use)}

\subsection{Mental Health \& Therapeutic Dialogue}
\begin{itemize}[noitemsep,topsep=0pt]
    \item Integrate into DBT, ACT, or narrative therapy contexts.
    \item Use it to externalize inner conflict and conflicting desires.
    \item Normalize self-doubt or avoidance as part of the SAGE map.
\end{itemize}

\subsection{Conflict Mediation or Repair}
\begin{itemize}[noitemsep,topsep=0pt]
    \item Use with dyads or teams to unpack both perspectives in grounded, safe terms.
    \item Apply CROWS to map others' contributions to the situation (past, hopes, expectations).
    \item Apply \textbf{FIND the LENS} to reduce misinterpretations or emotional reactivity.
\end{itemize}

\subsection{Workshop \& Curriculum Design}
\begin{itemize}[noitemsep,topsep=0pt]
    \item Design personal development workshops or group training around each Gnomy stage.
    \item Use it to teach leadership, communication, or emotional intelligence in schools or organizations.
    \item Create journaling prompts, guided meditations, or team debrief templates using each step.
\end{itemize}

---

\section{Why the Gnomy Framework Should Work: Theoretical Foundations}
The Gnomy Framework explicitly grounds its components in a range of established psychological and strategic theories, asserting that these foundations contribute to its efficacy. This theoretical alignment is a significant aspect of its design and potential impact. Where Gnomy extends or uniquely synthesizes these theories, it is explicitly noted.

\subsection{Mindfulness \& Grounding (FIND the LENS)}
DBT Mindfulness \cite{linehan1993cognitive} \& MBSR \cite{kabat1990full} provide strong evidence that grounding techniques like breathing and pausing improve emotional regulation, reduce anxiety, and enhance cognitive clarity. Gnomy's \textbf{FIND the LENS} sequence—integrating embodied awareness, goal mapping, and the nuanced discernment of stakes—mirrors widely used and validated techniques such as STOP and body scans in DBT, while adding a critical layer of multi-perspective emotional appraisal. The inclusion of 'Discern Stakes (and Divine Compassion)' in FIND, which introduces a "spiritual appraisal perspective" and "humble inquiry into who is experiencing the highest emotional or physical vulnerability," represents a conceptual extension beyond traditional Lazarusian appraisal theory \cite{lazarus1984stress}. While potentially broadening the framework's scope to existential concerns, this unique addition aims to cultivate empathy and align with an understanding of interconnected well-being.

\subsection{Narrative \& Externalization (LUX + CROWS)}
Psychological research and therapeutic models such as Narrative Therapy \cite{white1990narrative}, Internal Family Systems (IFS) \cite{schwartz1995internal}, and Cognitive Behavioral Therapy (CBT) \cite{beck1979cognitive} support the idea that naming emotions and constructing coherent narratives helps reduce distress and increase meaning. The \textbf{LUX} step aligns with these methods by encouraging emotional labeling, meaning-making, and introspective clarity through the lens of longing, usefulness, and friction. Its integration of neurochemistry (Dopamine, Serotonin, Noradrenaline) with the Kano model's Attractive, Performance, and Must-Be features \cite{kano1984attractive} is a highly innovative, cross-disciplinary approach, suggesting a physiological basis for emotional responses tied to perceived "features" of a situation. The \textbf{CROWS} stage brings in elements of mentalization \cite{fonagy2002affect}, perspective-taking, and attachment theory \cite{bowlby1969attachment}—all essential for emotional intelligence and relationship repair. Reflecting on the role of others (real, imagined, past, future) creates space for compassion, re-interpretation, and systemic insight.

\subsection{Resource-Oriented Thinking (SPARK)}
\textbf{SPARK} mirrors concepts from solution-focused brief therapy \cite{de1997building}, positive psychology \cite{seligman2011flourish}, and resilience theory \cite{masten2001ordinary}. Shifting attention toward resources, relationships, and internal strengths has been shown to increase optimism, motivation, and problem-solving capacity. It also encourages autonomy and competence—two key components of self-determination theory \cite{ryan2000self}.

\subsection{Risk Mapping \& Action Planning (SAGE)}
The \textbf{SAGE} component integrates strategic foresight/scenario planning \cite{schoemaker1995scenario}, stress inoculation training \cite{meichenbaum1985stress}, and values-based decision-making. By identifying outcome spectrums (Scary—Sufficient—Successful) and categorizing strategies (Avoid—Anchor—Aspire), users are better equipped to evaluate trade-offs, mitigate anxiety, and act intentionally. It also reflects ACT (Acceptance and Commitment Therapy) principles by balancing risk acceptance with committed action \cite{hayes1999acceptance}, and the OODA loop (Observe—Orient—Decide—Act) from military and strategic planning \cite{boyd1986destruction}—creating a feedback loop that favors adaptive iteration over rigid execution.

\subsection{Iterative and Cyclical Design}
Unlike rigid frameworks, Gnomy is built for real-life messiness. Its looping structure (returning to Grounding, Narrative, or Others when stuck) aligns with how humans process uncertainty and change. This makes it ideal for personal development, emotional healing, or strategic leadership in unpredictable environments \cite{kolb1984experiential}.

---

\section{Suggested Use Cases by Profession}
\begin{tabular}{|p{0.15\textwidth}|p{0.8\textwidth}|}
    \hline
    \textbf{Role} & \textbf{Example Use} \\
    \hline
    \textbf{Coach} & Guide a client through a business pivot using \textbf{LUX} to understand underlying emotional drives and then SPARK $\rightarrow$ SAGE to reframe fears and strategize action. \\
    \textbf{Therapist} & Use CROWS to help a client understand how childhood dynamics shape current interpersonal reactions, and then \textbf{LUX} to explore the emotional longing, usefulness, and friction present in their current relationships. \\
    \textbf{Team Facilitator} & Run a post-project debrief using \textbf{FIND the LENS} and \textbf{LUX} to explore what worked emotionally and interpersonally, specifically addressing the team's shared longing, perceptions of usefulness, and points of friction, and understanding the different perspectives of stakeholders. \\
    \textbf{Startup Founder} & Use SPARK to inventory team strengths and brainstorm pivot paths, followed by SAGE to stress-test risk scenarios, informed by insights from \textbf{LUX} regarding the team's intrinsic motivations (longing) and potential pain points (friction), and using \textbf{FIND the LENS} to anticipate stakeholder reactions. \\
    \textbf{Educator} & Integrate the Gnomy framework into leadership or emotional intelligence curricula to foster reflective decision-making, encouraging students to use \textbf{LUX} to analyze the emotional components of complex problems and \textbf{FIND the LENS} to develop empathy for diverse viewpoints. \\
    \textbf{Mediator} & Apply CROWS and SAGE with conflicting parties to find shared history, individual concerns, and acceptable compromises, enhanced by using \textbf{LUX} to understand the underlying emotional drivers (longing, usefulness, friction) contributing to each party's perspective, and employing \textbf{FIND the LENS} to clarify the different appraisal of stakes from each side. \\
    \hline
\end{tabular}

---

\section{Summary: Why the World Needs Gnomy}
In an era of burnout, isolation, polarization, and rapid change, the Gnomy Framework gives people a structured, imaginative way to reconnect with their inner compass and with each other. It invites emotional honesty and strategic thinking into the same space—bridging feeling and action, pain and power, isolation and connection. It empowers people not just to cope, but to cultivate meaning, collaborate wisely, and lead with heart.

---

\begin{thebibliography}{99}
    \bibitem{linehan1993cognitive} Linehan, M. M. (1993). \textit{Cognitive-Behavioral Treatment of Borderline Personality Disorder}. Guilford Press.
    \bibitem{kabat1990full} Kabat-Zinn, J. (1990). \textit{Full Catastrophe Living: Using the Wisdom of Your Body and Mind to Face Stress, Pain, and Illness}. Delta.
    \bibitem{white1990narrative} White, M., \& Epston, D. (1990). \textit{Narrative Means to Therapeutic Ends}. W. W. Norton \& Company.
    \bibitem{schwartz1995internal} Schwartz, R. C. (1995). \textit{Internal Family Systems Therapy}. Guilford Press.
    \bibitem{beck1979cognitive} Beck, A. T., Rush, A. J., Shaw, B. F., \& Emery, G. (1979). \textit{Cognitive Therapy of Depression}. Guilford Press.
    \bibitem{fonagy2002affect} Fonagy, P., Gergely, G., Jurist, E. L., \& Target, M. (2002). \textit{Affect Regulation, Mentalization, and the Development of the Self}. Other Press.
    \bibitem{bowlby1969attachment} Bowlby, J. (1969). \textit{Attachment and Loss, Vol. 1: Attachment}. Attachment and Loss.
    \bibitem{de1997building} de Shazer, S. (1997). \textit{Words Were Originally Magic}. W. W. Norton \& Company.
    \bibitem{seligman2011flourish} Seligman, M. E. P. (2011). \textit{Flourish: A Visionary New Understanding of Happiness and Well-being}. Free Press.
    \bibitem{ryan2000self} Ryan, R. M., \& Deci, E. L. (2000). Self-determination theory and the facilitation of intrinsic motivation, social development, and well-being. \textit{American Psychologist}, 55(1), 68–78.
    \bibitem{schoemaker1995scenario} Schoemaker, P. J. H. (1995). Scenario Planning: A Tool for Strategic Thinking. \textit{Sloan Management Review}, 36(2), 25–40.
    \bibitem{meichenbaum1985stress} Meichenbaum, D. (1985). \textit{Stress Inoculation Training}. Pergamon Press.
    \bibitem{hayes1999acceptance} Hayes, S. C., Strosahl, K. D., \& Wilson, K. G. (1999). \textit{Acceptance and Commitment Therapy: An Experiential Approach to Behavior Change}. Guilford Press.
    \bibitem{boyd1986destruction} Boyd, J. R. (1986). \textit{Destruction and Creation}. U.S. Army Command and General Staff College.
    \bibitem{kolb1984experiential} Kolb, D. A. (1984). \textit{Experiential Learning: Experience as the Source of Learning and Development}. Prentice-Hall.
    \bibitem{lazarus1984stress} Lazarus, R. S., \& Folkman, S. (1984). \textit{Stress, Appraisal, and Coping}. Springer Publishing Company.
    \bibitem{kano1984attractive} Kano, N., Seraku, N., Takahashi, F., \& Tsuji, S. (1984). Attractive quality and must-be quality. \textit{Journal of the Japanese Society for Quality Control}, 14(2), 39-48.
    \bibitem{masten2001ordinary} Masten, A. S. (2001). Ordinary magic: Resilience in development. \textit{American Psychologist}, 56(3), 227–238.
\end{thebibliography}

\end{document}
