The Gnomy Framework 1.
By Rickard V. Hultgren: Cultivating Emotional & Strategic Clarity
Creative Commons Attribution 4.0 International License (CC BY 4.0)
This work, ”The Gnomy Framework version 1.0: Cultivating Emotional & Strategic Clarity,” by Rickard
V. Hultgren, is licensed under a Creative Commons Attribution 4.0 International License. To view a copy
of this license, visithttp://creativecommons.org/licenses/by/4.0/.
LM
The Gnomy Framework is a structured yet flexible tool designed to enhance emotional self-leadership,
coaching, reflection, and narrative transformation. It’s particularly powerful when navigating complexity,
change, or interpersonal dynamics. Here’s a clear breakdown of its purpose and applications, categorized
by intent and user contexts.
— —

1.
Core Purpose of the Gnomy Framework
The Gnomy Framework provides a structured journey for individuals and groups to:

Anchorthemselves before reacting or making decisions.
Unravelinternal narratives and external influences.
Envisionfuture possibilities with emotional intelligence.
Strategizerisks, motivations, and responsibilities.
Iterateand evolve.
It supports a holistic process of emotional insight, cognitive clarity, and personal strategy-making. It’s an
ideal tool for coaching, personal growth, mental health support, group reflection, and leadership training.
—
2.
What Gnomy Should Be Used For
2.1. Coaching & Mentorship
To help clients:

Gain clarity in emotionally charged or complex situations.
Shift from rumination to structured insight.
Reframe failures, risks, or uncertainties into growth paths.
Clarify responsibilities and action options in relational contexts.
2.2. Self-Reflection & Emotional Navigation
For individuals to:

Understand how personal stories and social influences shape decisions.
Ground themselves during stress or at key decision points.
Identify emotional needs, risks, and motivations.
Develop resilience through narrative self-work and emotional agency.
2.3. Team & Group Dynamics
For use in teams to:

Reflect on past collaboration patterns and improve communication.
Clarify individual and shared goals, risks, and unspoken assumptions.
Create psychological safety by mapping different emotional truths.
Balance personal contributions with team-wide needs.
2.4. Leadership & Strategy Design
To support leaders in:

Balancing rational planning with underlying emotional currents.
Anticipating team member perspectives and motivations.
Making decisions aligned with both vision (aspiration) and risk awareness.
Creating roadmaps that account for psychological barriers and enablers.
—
3.
What Gnomy Could Be Used For
(Extensions and Creative Use)
3.1. Mental Health & Therapeutic Dialogue
Integrate into DBT, ACT, or narrative therapy contexts.
Use it to externalize inner conflict and conflicting desires.
Normalize self-doubt or avoidance as part of the SAGE map.
3.2. Gamified Self-Help or Learning Tools
Build a mobile game or app (e.g., “Gnomy the Garden Guide”) around its steps.
Use metaphorical trees, lanterns, crows, and sparks to make emotional reflection engaging.
Turn SAGE into a quest system that builds resilience and emotional literacy.
3.3. Conflict Mediation or Repair
Use with dyads or teams to unpack both perspectives in grounded, safe terms.
Apply CROWS to map others’ contributions to the situation (past, hopes, expectations).
Apply FIND to reduce misinterpretations or emotional reactivity.
3.4. Workshop & Curriculum Design
Design personal development workshops or group training around each Gnomy stage.
Use it to teach leadership, communication, or emotional intelligence in schools or organizations.
Create journaling prompts, guided meditations, or team debrief templates using each step.
—
4.
Summary of Gnomy Component Roles
Stage Purpose Function
G – Grounding Establish presence and perspective. Breath, pause, curiosity, perspective
framing.
N – Narrative Understand internal drives and ten-
sions.
Identify pain points and feelings to-
wards the challenge using theLUX
Lantern model.
O – Others Map relational influences and shared
identity.
Understand self and others across
time.
M – More Identify inner and outer resources. Leverage strengths, relationships,
knowledge.
Y – Yikes! Strategize through risks and change. Assess risk spectrum, derive action,
prioritize, evolve.
—
5.
In Short: The Gnomy Framework
The Gnomy Framework is a compass for emotional and strategic clarity. It helps people slow down,
reflect deeply, and move forward wisely. It belongs in the toolkit of coaches, creators, leaders, therapists,
facilitators, and reflective individuals seeking to bridge emotion and action with intelligence and care.

Imagine this: A Gnomish Journey of Pond-Bound Reflection. A small gnome stands on a delicate, hybrid
plant – perhaps a blend of strawberry and water lily – nestled within a tranquil pond. This pond embodies
a specific context, while the individual plant represents a distinct situation. Plants in this pond are
interconnected by slender tendrils.
In their left hand, the gnome holds a lantern of current feeling. From bottles in their belt, filled with
luminescent powder of different colours, the gnome pours some powder into the lantern, and it begins to
glow. This represents theLUX Lantern modelof their inner narrative.
On the right shoulder, a black crow perches confidently, ready to fly to other gnomes with messages.
This embodiesCROWS, understanding how others shape their story.
In their left hand, the gnome juggles a glowing spark, catching it with practiced joy. This spark floats
in the air, manipulated by the gnome’s twisting hand, creating a gentle wind that gathers dried leaves
from the gnome’s leaf-heap. This symbolizesSPARK, uncovering resources and possibilities.
In their right hand, the gnome holds a single sage leaf, dry and bitter on the tongue. It’s a reminder:
wisdom isn’t always sweet. It comes with limits, trade-offs, and hard-earned truths. This represents
SAGE, facing risks and choosing wisely.
And so, this gnome walks – leaving footprints shaped by emotion, relationship, strength, and restraint.
Every step is a journey through GNOMY: Grounding, Narrative, Others, More, and Yikes! (Reflection,
Story, Others, Courage, and Change).
—

6.
The Gnomy Framework: Your Step-By-Step
Guide
6.1. Step 0: Grounding (FIND) — Find Your Inner Position
FIND helps you regain perspective by pausing to explore goals, emotions, and responsibility with intention.
Before diving into analysis or action, ground yourself in the present moment to regain perspective.
Grounding prepares your mind and body for reflective, intentional thinking. Use theFINDsequence to
anchor yourself:

Feel your breath: Breathe in for 4 counts, hold for 4. This regulates your nervous system and
slows racing thoughts. F – Feel your breath/urge Embodied grounding Helps interrupt reactivity
and build emotional awareness before analysis.
Introduce pause: Silently say, “It’s just in my mind. What else can I find?” Goal-mapping and
perspective expansion. Surfaces implicit motivations and helps distinguish between short-term urges
and deeper goals.
Notice urgency: Action-orientation + systems thinking Asks what can be done constructively,
not just what is felt. Encourages agency over rumination.
Dismantle assumptions: Narrative humility, shared responsibility Avoids blame, embraces
complexity, promotes collective insight and equity thinking.
6.2. Step 1: Narrative of Emotion and Meaning
6.2.1 1.1 Narrative (LUX) – Explore Your Emotions and Their Roots
Emotions are signals, not noise. They reflect unmet needs, broken expectations, or meaningful desires. The
LUX Lantern modelhelps you dive into your emotional narrative, illuminating your inner landscape
through three key axes:
L – Longing (Dopamine×Kano’s Attractive Features): What desires, hopes, or unspoken
rewards are present? What do you secretly long for in this situation? This speaks to your intrinsic
motivations and the “delighters” you seek.
U – Usefulness (Serotonin×Kano’s Performance Features): How does this situation
impact your sense of reliability, control, well-being, or self-worth? What aspects feel reliable or are
providing a sense of competence? This focuses on the functional value and performance aspects
that contribute to your satisfaction.
X – X-Friction (Noradrenaline×Kano’s Must-Be/Reverse Features): What threats,
irritations, pressures, or unmet basic needs are you experiencing? What feels like an unavoidable
obstacle or something that, if absent, would cause significant dissatisfaction? This addresses the
“must-be” qualities and sources of negative affect.
6.2.2 1.2 Others (CROWS) – Who Else Is Part of This Emotional Story?
We are shaped by relationships—past, present, and imagined. A crow will fly to other gnomes, inviting
them to transfer their insights to your understanding. AskCROWSto place your experience in a
broader social context:

Current others: Who else is involved now? How do their perspectives or emotions interact with
yours?
Retrospective self: How would your past self interpret or feel about this?
Optional future self: What emotional legacy are you creating? What might your future self
think or feel?
Wished-for others: Are there people you hope to connect with or impress? Fantasies, role models,
or longings?
Shared past: What shared experiences or histories are influencing this moment?
6.3. Step 2: Reframe Through Shared Competence and Action
6.3.1 2.1 More (SPARK) – Uncover Resources, Strengths, and Options
Imagine the gnome’s current situation as a plant in a pond. To move to another ”situation” (represented
by a different plant), the gnome needs a leaf boat—an old, dried leaf from their personal leaf heap. The
gnome steers a floating spark in the air by twisting and gesturing with their left hand. This spark, when
moved circularly, heats the surrounding air, creating a gentle wind. With this wind, the gnome directs a
dried leaf from their heap to the water’s edge of their current plant. Stepping onto this leaf, the gnome
then uses it as a cable ferry, pulling along a tendril stretched between the plants to cross to the next.
This leaf boat is the asset that enables the gnome to navigate from one situation to another.
From this awareness, you can move towards possibility. Expand your narrative understanding through
the wisdom of other ”gnomes” (people or external knowledge) or your own inner resources. Use SPARK
to inventory the personal and interpersonal resources available to you:
Strengths: What abilities, traits, or skills do you bring to this moment?
Possibilities: What courses of action exist? Be creative—don’t filter options yet.
Assets: What tangible and intangible resources (e.g., tools, time, information, shared community
resources) can you rely on?
Resourced Relationships: Who can walk with you in this? Who offers strength, support,
challenge, or insight?
Knowledge & Experience: What lessons have you already learned? What frameworks or facts
do you already hold?
6.3.2 2.2 Yikes! (SAGE) – Face Risks, Limits, and Priorities

On the tendril there are sage leaves growing, symbolizing the wisdom and often bitter truths involved in
facing risks. Before committing to action, reflect on risk, responsibility, and refinement. UseSAGEto
evaluate outcomes and adapt.

Spectrum: Define three kinds of outcomes:
Scary: What’s the worst-case scenario? What would feel like a failure?
Sufficient: What’s “good enough” to move forward?
Successful: What would a meaningful win look like?
Actions: For each outcome, what are possible reactions or strategies?
Avoidance: What might I try to escape or delay?
Anchoring: What stable actions keep me grounded under stress?
Aspiration: What bold or creative steps move me toward success?
Gauge: Assess urgency and importance.
Evolution: The world changes—and so do you.
Examine: What feedback are you receiving?
Evaluate: What’s working, what’s not?
Edit: What needs adjusting or dropping?
Expand: Connect to the pain-points. Reflect on what changed through your Gnomy process.
What pain points were resolved? What challenges remain or new tensions arose? Who
benefited, who was affected, and what became clearer? Share this with those impacted by the
results. This step isn’t just closure—it’s a bridge to the next cycle, grounding future action in
real insight and shared learning.
6.4. Feedback Loop
If your plan feels shaky, your emotions shift, or new obstacles appear, it’s like the gnome realizing their
current plant isn’t stable or the path forward is unclear. In such moments, the gnome doesn’t despair;
instead, they might return toStep 0 (Grounding)to steady their footing on their current plant, or
revisitStep 1 (Narrative)to re-evaluate the story of where they are and where they want to go. The
Gnomy Framework, much like the interconnected plants in the pond, is cyclical and flexible—designed for
ongoing clarity, growth, and recalibration as you navigate from one ”situation-plant” to the next. —

7.
Why the Gnomy Framework Should Work
7.1. Mindfulness & Grounding (FIND)
DBT Mindfulness & MBSR provide strong evidence that grounding techniques like breathing and pausing
improve emotional regulation, reduce anxiety, and enhance cognitive clarity [ 1 , 2 ]. Gnomy’s FIND
sequence—“Feel breath, Introduce pause, Notice curiosity, Define perspective”—mirrors widely used and
validated techniques such as STOP and body scans in DBT.

7.2. Narrative & Externalization (LUX + CROWS)
Psychological research and therapeutic models such as Narrative Therapy [ 3 ], Internal Family Systems
(IFS) [ 4 ], and Cognitive Behavioral Therapy (CBT) [ 5 ] support the idea that naming emotions and
constructing coherent narratives helps reduce distress and increase meaning. TheLUXstep aligns with
these methods by encouraging emotional labeling, meaning-making, and introspective clarity through
the lens of longing, usefulness, and friction. The CROWS stage brings in elements of mentalization [ 6 ],
perspective-taking, and attachment theory [ 7 ]—all essential for emotional intelligence and relationship
repair. Reflecting on the role of others (real, imagined, past, future) creates space for compassion,
re-interpretation, and systemic insight.

7.3. Resource-Oriented Thinking (SPARK)
SPARK mirrors concepts from solution-focused brief therapy [ 8 ], positive psychology [ 9 ], and resilience
theory. Shifting attention toward resources, relationships, and internal strengths has been shown to increase
optimism, motivation, and problem-solving capacity. It also encourages autonomy and competence—two
key components of self-determination theory [10].
7.4. Risk Mapping & Action Planning (SAGE)
The SAGE component integrates strategic foresight/scenario planning [ 11 ], stress inoculation training [ 12 ],
and values-based decision-making. By identifying outcome spectrums (Scary–Sufficient–Successful) and
categorizing strategies (Avoid–Anchor–Aspire), users are better equipped to evaluate trade-offs, mitigate
anxiety, and act intentionally. It also reflects ACT (Acceptance and Commitment Therapy) principles by
balancing risk acceptance with committed action [ 13 ], and the OODA loop (Observe–Orient–Decide–Act)
from military and strategic planning [ 14 ]—creating a feedback loop that favors adaptive iteration over
rigid execution.
7.5. Iterative and Cyclical Design
Unlike rigid frameworks, Gnomy is built for real-life messiness. Its looping structure (returning to
Grounding, Narrative, or Others when stuck) aligns with how humans process uncertainty and change.
This makes it ideal for personal development, emotional healing, or strategic leadership in unpredictable
environments [15].
—
8.
Suggested Use Cases by Profession
Role Example Use
Coach Guide a client through a business pivot usingLUXto understand underlying
emotional drives and then SPARK→SAGE to reframe fears and strategize action.
Therapist Use CROWS to help a client understand how childhood dynamics shape current
interpersonal reactions, and thenLUXto explore the emotional longing, usefulness,
and friction present in their current relationships.
Team Facilita-
tor
Run a post-project debrief using FIND andLUXto explore what worked emotionally
and interpersonally, specifically addressing the team’s shared longing, perceptions of
usefulness, and points of friction.
Startup
Founder
Use SPARK to inventory team strengths and brainstorm pivot paths, followed by
SAGE to stress-test risk scenarios, informed by insights fromLUXregarding the
team’s intrinsic motivations (longing) and potential pain points (friction).
Educator Integrate the Gnomy framework into leadership or emotional intelligence curricula
to foster reflective decision-making, encouraging students to useLUXto analyze
the emotional components of complex problems.
Mediator Apply CROWS and SAGE with conflicting parties to find shared history, individual
concerns, and acceptable compromises, enhanced by usingLUXto understand
the underlying emotional drivers (longing, usefulness, friction) contributing to each
party’s perspective.
—
9.
Summary: Why the World Needs Gnomy
In an era of burnout, isolation, polarization, and rapid change, the Gnomy Framework gives people
a structured, imaginative way to reconnect with their inner compass and with each other. It invites
emotional honesty and strategic thinking into the same space—bridging feeling and action, pain and power,
isolation and connection. It empowers people not just to cope, but to cultivate meaning, collaborate
wisely, and lead with heart.
—

References
[1]Linehan, M. M. (1993).Cognitive-Behavioral Treatment of Borderline Personality Disorder. Guilford
Press.
[2]Kabat-Zinn, J. (1990).Full Catastrophe Living: Using the Wisdom of Your Body and Mind to Face
Stress, Pain, and Illness. Delta.
[3] White, M., Epston, D. (1990).Narrative Means to Therapeutic Ends. W. W. Norton & Company.
[4] Schwartz, R. C. (1995).Internal Family Systems Therapy. Guilford Press.
[5]Beck, A. T., Rush, A. J., Shaw, B. F., Emery, G. (1979).Cognitive Therapy of Depression. Guilford
Press.
[6]Fonagy, P., Gergely, G., Jurist, E. L., Target, M. (2002).Affect Regulation, Mentalization, and the
Development of the Self. Other Press.
[7] Bowlby, J. (1969).Attachment and Loss, Vol. 1: Attachment. Attachment and Loss.
[8]de Shazer, S., Berg, I. K. (1997).Building Solutions in Brief Therapy. W. W. Norton & Company.
[9]Seligman, M. E. P. (2011).Flourish: A Visionary New Understanding of Happiness and Well-being.
Free Press.
[10]Ryan, R. M., Deci, E. L. (2000). Self-determination theory and the facilitation of intrinsic motivation,
social development, and well-being.American Psychologist, 55(1), 68–78.

[11]Schoemaker, P. J. H. (1995). Scenario planning: A tool for strategic thinking.Sloan Management
Review, 36(2), 25–40.

[12] Meichenbaum, D. (1985).Stress Inoculation Training. Pergamon Press.

[13]Hayes, S. C., Strosahl, K. D., Wilson, K. G. (1999).Acceptance and Commitment Therapy: An
Experiential Approach to Behavior Change. Guilford Press.

[14] Boyd, J. R. (1986).Destruction and Creation. Unpublished manuscript.

[15]Kolb, D. A. (1984).Experiential Learning: Experience as the Source of Learning and Development.
Prentice-Hall.
