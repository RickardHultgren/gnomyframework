\documentclass{article}
\usepackage[a4paper, margin=1in]{geometry} % More concise geometry
\usepackage{ragged2e} % For justified text
\usepackage{enumitem} % For custom list item spacing
\usepackage{fontawesome5} % For icons like compass
\usepackage{hyperref} % For hyperlinks
\usepackage[svgnames]{xcolor} % For more color options
\usepackage{titlesec} % For custom section formatting
\usepackage{tikz} % For drawing the gnome illustration (if you include it later)
\usepackage[most]{tcolorbox} % For beautiful boxes
\usepackage{microtype} % For better typography (font protrusion, character tracking etc.)

% Define Victorian-inspired colors
\definecolor{VictorianDeepGreen}{HTML}{2F4F4F} % Dark Slate Gray
\definecolor{VictorianPlum}{HTML}{673147} % Muted Plum/Deep Rose
\definecolor{VictorianTeal}{HTML}{008080} % Teal
\definecolor{VictorianGold}{HTML}{B8860B} % Dark Goldenrod
\definecolor{VictorianBrown}{HTML}{8B4513} % Saddle Brown
\definecolor{VictorianGrey}{HTML}{4F4F4F} % Dark Grey
\definecolor{VictorianCream}{HTML}{F5F5DC} % Beige

% Creative Commons License (CC BY 4.0)
% Using tcolorbox for a visually appealing license notice
\tcbset{
  licensebox/.style={
    colback=VictorianCream, % Creamy background
    colframe=VictorianGrey, % Dark Grey frame
    boxsep=5pt, % Space between content and box
    arc=4pt, % Rounded corners
    boxrule=0.5pt, % Thin border
    left=10pt,right=10pt,top=5pt,bottom=5pt, % Padding
    fonttitle=\bfseries,
    fontupper=\small,
    halign title=center,
    valign=middle,
    nobeforeafterifvbox
  }
}

\newcommand{\printgnomylicense}{%
\begin{tcolorbox}[licensebox]
  \centering
  \textbf{Creative Commons Attribution 4.0 International License (CC BY 4.0)} \\
  This work, "The Gnomy Framework version 1.0: Cultivating Emotional \& Strategic Clarity," by Rickard V. Hultgren, is licensed under a Creative Commons Attribution 4.0 International License. To view a copy of this license, visit \url{http://creativecommons.org/licenses/by/4.0/}. \\
  \faIcon{creative-commons} \faIcon{creative-commons-by}
\end{tcolorbox}
}

% Custom section formatting with more visual flair using Victorian colors
\titleformat{\section}[block]{\centering\color{VictorianDeepGreen}\huge\bfseries}{\thesection.}{1em}{\color{VictorianTeal}\rule{\linewidth}{0.8pt}\\[1ex]}
\titlespacing*{\section}{0pt}{3.5ex plus 1ex minus .2ex}{2.3ex plus .2ex}

\titleformat{\subsection}[hang]{\color{VictorianPlum}\Large\bfseries}{\thesubsection.}{1em}{}
\titlespacing*{\subsection}{0pt}{3.25ex plus 1ex minus .2ex}{1.5ex plus .2ex}

\begin{document}

\begin{center}
    \vspace*{1em} % Add some vertical space
    \color{VictorianBrown}\textbf{\Huge The Gnomy Framework \version{1.0}} \\
    \vspace{0.5em} % Space between title and author
    \color{VictorianGrey}\large By Rickard V. Hultgren: Cultivating Emotional \& Strategic Clarity
    \vspace{1em} % Space after author line
\end{center}

\printgnomylicense % Print the custom license box

\vspace{1em} % Space after the license box

\justify % Ensures the text is justified
The Gnomy Framework is a structured yet flexible tool designed to enhance emotional self-leadership, coaching, reflection, and narrative transformation. It's particularly powerful when navigating complexity, change, or interpersonal dynamics. Here's a clear breakdown of its purpose and applications, categorized by intent and user contexts.

---
---

\section{Core Purpose of the Gnomy Framework}
Gnomy helps individuals and groups to:
\begin{itemize}[noitemsep,topsep=0pt]
    \item \textbf{Ground} themselves before reacting or making decisions.
    \item \textbf{Clarify} internal narratives and external influences.
    \item \textbf{Explore} future possibilities with emotional intelligence.
    \item \textbf{Map} risks, motivations, and responsibilities.
    \item \textbf{Reflect} and evolve through structured iteration.
\end{itemize}
It supports a holistic process of emotional insight, cognitive clarity, and personal strategy-making. It's an ideal tool for coaching, personal growth, mental health support, group reflection, and leadership training.

---

\section{What Gnomy Should Be Used For}

\subsection{Coaching \& Mentorship}
To help clients:
\begin{itemize}[noitemsep,topsep=0pt]
    \item Gain clarity in emotionally charged or complex situations.
    \item Shift from rumination to structured insight.
    \item Reframe failures, risks, or uncertainties into growth paths.
    \item Clarify responsibilities and action options in relational contexts.
\end{itemize}

\subsection{Self-Reflection \& Emotional Navigation}
For individuals to:
\begin{itemize}[noitemsep,topsep=0pt]
    \item Understand how personal stories and social influences shape decisions.
    \item Ground themselves during stress or at key decision points.
    \item Identify emotional needs, risks, and motivations.
    \item Develop resilience through narrative self-work and emotional agency.
\end{itemize}

\subsection{Team \& Group Dynamics}
For use in teams to:
\begin{itemize}[noitemsep,topsep=0pt]
    \item Reflect on past collaboration patterns and improve communication.
    \item Clarify individual and shared goals, risks, and unspoken assumptions.
    \item Create psychological safety by mapping different emotional truths.
    \item Balance personal contributions with team-wide needs.
\end{itemize}

\subsection{Leadership \& Strategy Design}
To support leaders in:
\begin{itemize}[noitemsep,topsep=0pt]
    \item Balancing rational planning with underlying emotional currents.
    \item Anticipating team member perspectives and motivations.
    \item Making decisions aligned with both vision (aspiration) and risk awareness.
    \item Creating roadmaps that account for psychological barriers and enablers.
\end{itemize}

---

\section{What Gnomy Could Be Used For (Extensions and Creative Use)}

\subsection{Mental Health \& Therapeutic Dialogue}
\begin{itemize}[noitemsep,topsep=0pt]
    \item Integrate into DBT, ACT, or narrative therapy contexts.
    \item Use it to externalize inner conflict and conflicting desires.
    \item Normalize self-doubt or avoidance as part of the SAGE map.
\end{itemize}

\subsection{Gamified Self-Help or Learning Tools}
\begin{itemize}[noitemsep,topsep=0pt]
    \item Build a mobile game or app (e.g., ``Gnomy the Garden Guide'') around its steps.
    \item Use metaphorical trees, lanterns, crows, and sparks to make emotional reflection engaging.
    \item Turn SAGE into a quest system that builds resilience and emotional literacy.
\end{itemize}

\subsection{Conflict Mediation or Repair}
\begin{itemize}[noitemsep,topsep=0pt]
    \item Use with dyads or teams to unpack both perspectives in grounded, safe terms.
    \item Apply CROWS to map others' contributions to the situation (past, hopes, expectations).
    \item Apply FIND to reduce misinterpretations or emotional reactivity.
\end{itemize}

\subsection{Workshop \& Curriculum Design}
\begin{itemize}[noitemsep,topsep=0pt]
    \item Design personal development workshops or group training around each Gnomy stage.
    \item Use it to teach leadership, communication, or emotional intelligence in schools or organizations.
    \item Create journaling prompts, guided meditations, or team debrief templates using each step.
\end{itemize}

---

\section{Summary of Gnomy Component Roles}
\begin{tabular}{|p{0.2\textwidth}|p{0.35\textwidth}|p{0.35\textwidth}|}
    \hline
    \textbf{Stage} & \textbf{Purpose} & \textbf{Function} \\
    \hline
    \textbf{G – Grounding} & Establish presence and perspective. & Breath, pause, curiosity, perspective framing. \\
    \textbf{N – Narrative} & Understand internal drives and tensions. & Identify pain points and feelings towards the challenge using the \textbf{LUX Lantern model}. \\
    \textbf{O – Others} & Map relational influences and shared identity. & Understand self and others across time. \\
    \textbf{M – More} & Identify inner and outer resources. & Leverage strengths, relationships, knowledge. \\
    \textbf{Y – Yikes!} & Strategize through risks and change. & Assess risk spectrum, derive action, prioritize, evolve. \\
    \hline
\end{tabular}

---

\section{In Short: The Gnomy Framework}
The Gnomy Framework is a compass for emotional and strategic clarity. It helps people slow down, reflect deeply, and move forward wisely. It belongs in the toolkit of coaches, creators, leaders, therapists, facilitators, and reflective individuals seeking to bridge emotion and action with intelligence and care.
\begin{center}
\includegraphics[width=0.8\linewidth]{GNOMY.png}
\end{center}
Imagine this: A Gnomish Journey of Pond-Bound Reflection. A small gnome stands on a delicate, hybrid plant – perhaps a blend of strawberry and water lily – nestled within a tranquil pond. This pond embodies a specific context, while the individual plant represents a distinct situation. Plants in this pond are interconnected by slender tendrils.

In their left hand, the gnome holds a lantern of current feeling. From bottles in their belt, filled with luminescent powder of different colours, the gnome pours some powder into the lantern, and it begins to glow. This represents the \textbf{LUX Lantern model} of their inner narrative.

On the right shoulder, a black crow perches confidently, ready to fly to other gnomes with messages. This embodies \textbf{CROWS}, understanding how others shape their story.

In their left hand, the gnome juggles a glowing spark, catching it with practiced joy. This spark floats in the air, manipulated by the gnome's twisting hand, creating a gentle wind that gathers dried leaves from the gnome's leaf-heap. This symbolizes \textbf{SPARK}, uncovering resources and possibilities.

In their right hand, the gnome holds a single sage leaf, dry and bitter on the tongue. It's a reminder: wisdom isn't always sweet. It comes with limits, trade-offs, and hard-earned truths. This represents \textbf{SAGE}, facing risks and choosing wisely.

And so, this gnome walks -- leaving footprints shaped by emotion, relationship, strength, and restraint. Every step is a journey through GNOMY: Grounding, Narrative, Others, More, and Yikes! (Reflection, Story, Others, Courage, and Change).

---

\section{The Gnomy Framework: Your Step-By-Step Guide}

\subsection{Step 0: Grounding (FIND) — Find Your Inner Position}
FIND helps you regain perspective by pausing to explore goals, emotions, and responsibility with intention.

Before diving into analysis or action, ground yourself in the present moment to regain perspective. Grounding prepares your mind and body for reflective, intentional thinking. Use the \textbf{FIND} sequence to anchor yourself:
\begin{itemize}[noitemsep,topsep=0pt]
    \item \textbf{Feel your breath}: Breathe in for 4 counts, hold for 4. This regulates your nervous system and slows racing thoughts. F – Feel your breath/urge	Embodied grounding	Helps interrupt reactivity and build emotional awareness before analysis.
    \item \textbf{Introduce pause}: Silently say, ``It's just in my mind. What else can I find?'' Goal-mapping and perspective expansion. Surfaces implicit motivations and helps distinguish between short-term urges and deeper goals.
    \item \textbf{Notice urgency}: Action-orientation + systems thinking	Asks what can be done constructively, not just what is felt. Encourages agency over rumination.
    \item \textbf{ Dismantle assumptions}: Narrative humility, shared responsibility	Avoids blame, embraces complexity, promotes collective insight and equity thinking.
\end{itemize}

\subsection{Step 1: Narrative of Emotion and Meaning}

\subsubsection{1.1 Narrative (LUX) – Explore Your Emotions and Their Roots}
Emotions are signals, not noise. They reflect unmet needs, broken expectations, or meaningful desires. The \textbf{LUX Lantern model} helps you dive into your emotional narrative, illuminating your inner landscape through three key axes:
\begin{itemize}[noitemsep,topsep=0pt]
    \item \textbf{L – Longing (Dopamine $\times$ Kano’s Attractive Features)}: What desires, hopes, or unspoken rewards are present? What do you secretly long for in this situation? This speaks to your intrinsic motivations and the ``delighters'' you seek.
    \item \textbf{U – Usefulness (Serotonin $\times$ Kano’s Performance Features)}: How does this situation impact your sense of reliability, control, well-being, or self-worth? What aspects feel reliable or are providing a sense of competence? This focuses on the functional value and performance aspects that contribute to your satisfaction.
    \item \textbf{X – X-Friction (Noradrenaline $\times$ Kano’s Must-Be/Reverse Features)}: What threats, irritations, pressures, or unmet basic needs are you experiencing? What feels like an unavoidable obstacle or something that, if absent, would cause significant dissatisfaction? This addresses the ``must-be'' qualities and sources of negative affect.
\end{itemize}

\subsubsection{1.2 Others (CROWS) – Who Else Is Part of This Emotional Story?}
We are shaped by relationships—past, present, and imagined. A crow will fly to other gnomes, inviting them to transfer their insights to your understanding. Ask \textbf{CROWS} to place your experience in a broader social context:
\begin{itemize}[noitemsep,topsep=0pt]
    \item \textbf{Current others}: Who else is involved now? How do their perspectives or emotions interact with yours?
    \item \textbf{Retrospective self}: How would your past self interpret or feel about this?
    \item \textbf{Optional future self}: What emotional legacy are you creating? What might your future self think or feel?
    \item \textbf{Wished-for others}: Are there people you hope to connect with or impress? Fantasies, role models, or longings?
    \item \textbf{Shared past}: What shared experiences or histories are influencing this moment?
\end{itemize}

\subsection{Step 2: Reframe Through Shared Competence and Action}

\subsubsection{2.1 More (SPARK) – Uncover Resources, Strengths, and Options}
Imagine the gnome's current situation as a plant in a pond. To move to another "situation" (represented by a different plant), the gnome needs a leaf boat—an old, dried leaf from their personal leaf heap. The gnome steers a floating spark in the air by twisting and gesturing with their left hand. This spark, when moved circularly, heats the surrounding air, creating a gentle wind. With this wind, the gnome directs a dried leaf from their heap to the water’s edge of their current plant. Stepping onto this leaf, the gnome then uses it as a cable ferry, pulling along a tendril stretched between the plants to cross to the next. This leaf boat is the asset that enables the gnome to navigate from one situation to another.

From this awareness, you can move towards possibility. Expand your narrative understanding through the wisdom of other "gnomes" (people or external knowledge) or your own inner resources. Use SPARK to inventory the personal and interpersonal resources available to you:
\begin{itemize}[noitemsep,topsep=0pt]
    \item \textbf{Strengths}: What abilities, traits, or skills do you bring to this moment?
    \item \textbf{Possibilities}: What courses of action exist? Be creative—don't filter options yet.
    \item \textbf{Assets}: What tangible and intangible resources (e.g., tools, time, information, shared community resources) can you rely on?
    \item \textbf{Resourced Relationships}: Who can walk with you in this? Who offers strength, support, challenge, or insight?
    \item \textbf{Knowledge \& Experience}: What lessons have you already learned? What frameworks or facts do you already hold?
\end{itemize}

\subsubsection{2.2 Yikes! (SAGE) – Face Risks, Limits, and Priorities}
On the tendril there are sage leaves growing, symbolizing the wisdom and often bitter truths involved in facing risks. Before committing to action, reflect on risk, responsibility, and refinement. Use \textbf{SAGE} to evaluate outcomes and adapt.
\begin{itemize}[noitemsep,topsep=0pt]
    \item \textbf{Spectrum}: Define three kinds of outcomes:
    \begin{itemize}[noitemsep,topsep=0pt]
        \item \textbf{Scary}: What's the worst-case scenario? What would feel like a failure?
        \item \textbf{Sufficient}: What's ``good enough'' to move forward?
        \item \textbf{Successful}: What would a meaningful win look like?
    \end{itemize}
    \item \textbf{Actions}: For each outcome, what are possible reactions or strategies?
    \begin{itemize}[noitemsep,topsep=0pt]
        \item \textbf{Avoidance}: What might I try to escape or delay?
        \item \textbf{Anchoring}: What stable actions keep me grounded under stress?
        \item \textbf{Aspiration}: What bold or creative steps move me toward success?
    \end{itemize}
    \item \textbf{Gauge}: Assess urgency and importance.
    \item \textbf{Evolution}: The world changes—and so do you.
    \begin{itemize}[noitemsep,topsep=0pt]
        \item \textbf{Examine}: What feedback are you receiving?
        \item \textbf{Evaluate}: What's working, what's not?
        \item \textbf{Edit}: What needs adjusting or dropping?
        \item \textbf{Expand}: Connect to the pain-points. Reflect on what changed through your Gnomy process. What pain points were resolved? What challenges remain or new tensions arose? Who benefited, who was affected, and what became clearer? Share this with those impacted by the results. This step isn't just closure—it's a bridge to the next cycle, grounding future action in real insight and shared learning.
    \end{itemize}
\end{itemize}

\subsection{Feedback Loop}
If your plan feels shaky, your emotions shift, or new obstacles appear, it's like the gnome realizing their current plant isn't stable or the path forward is unclear. In such moments, the gnome doesn't despair; instead, they might return to \textbf{Step 0 (Grounding)} to steady their footing on their current plant, or revisit \textbf{Step 1 (Narrative)} to re-evaluate the story of where they are and where they want to go. The Gnomy Framework, much like the interconnected plants in the pond, is cyclical and flexible—designed for ongoing clarity, growth, and recalibration as you navigate from one "situation-plant" to the next.
---

\section{Why the Gnomy Framework Should Work}

\subsection{Mindfulness \& Grounding (FIND)}
DBT Mindfulness \& MBSR provide strong evidence that grounding techniques like breathing and pausing improve emotional regulation, reduce anxiety, and enhance cognitive clarity \cite{linehan1993cognitive, kabat1990full}. Gnomy's FIND sequence—``Feel breath, Introduce pause, Notice curiosity, Define perspective''—mirrors widely used and validated techniques such as STOP and body scans in DBT.

\subsection{Narrative \& Externalization (LUX + CROWS)}
Psychological research and therapeutic models such as Narrative Therapy \cite{white1990narrative}, Internal Family Systems (IFS) \cite{schwartz1995internal}, and Cognitive Behavioral Therapy (CBT) \cite{beck1979cognitive} support the idea that naming emotions and constructing coherent narratives helps reduce distress and increase meaning. The \textbf{LUX} step aligns with these methods by encouraging emotional labeling, meaning-making, and introspective clarity through the lens of longing, usefulness, and friction. The CROWS stage brings in elements of mentalization \cite{fonagy2002affect}, perspective-taking, and attachment theory \cite{bowlby1969attachment}—all essential for emotional intelligence and relationship repair. Reflecting on the role of others (real, imagined, past, future) creates space for compassion, re-interpretation, and systemic insight.

\subsection{Resource-Oriented Thinking (SPARK)}
SPARK mirrors concepts from solution-focused brief therapy \cite{de1997building}, positive psychology \cite{seligman2011flourish}, and resilience theory. Shifting attention toward resources, relationships, and internal strengths has been shown to increase optimism, motivation, and problem-solving capacity. It also encourages autonomy and competence—two key components of self-determination theory \cite{ryan2000self}.

\subsection{Risk Mapping \& Action Planning (SAGE)}
The SAGE component integrates strategic foresight/scenario planning \cite{schoemaker1995scenario}, stress inoculation training \cite{meichenbaum1985stress}, and values-based decision-making. By identifying outcome spectrums (Scary–Sufficient–Successful) and categorizing strategies (Avoid–Anchor–Aspire), users are better equipped to evaluate trade-offs, mitigate anxiety, and act intentionally. It also reflects ACT (Acceptance and Commitment Therapy) principles by balancing risk acceptance with committed action \cite{hayes1999acceptance}, and the OODA loop (Observe–Orient–Decide–Act) from military and strategic planning \cite{boyd1986destruction}—creating a feedback loop that favors adaptive iteration over rigid execution.

\subsection{Iterative and Cyclical Design}
Unlike rigid frameworks, Gnomy is built for real-life messiness. Its looping structure (returning to Grounding, Narrative, or Others when stuck) aligns with how humans process uncertainty and change. This makes it ideal for personal development, emotional healing, or strategic leadership in unpredictable environments \cite{kolb1984experiential}.

---

\section{Suggested Use Cases by Profession}
\begin{tabular}{|p{0.15\textwidth}|p{0.8\textwidth}|}
    \hline
    \textbf{Role} & \textbf{Example Use} \\
    \hline
    \textbf{Coach} & Guide a client through a business pivot using \textbf{LUX} to understand underlying emotional drives and then SPARK $\rightarrow$ SAGE to reframe fears and strategize action. \\
    \textbf{Therapist} & Use CROWS to help a client understand how childhood dynamics shape current interpersonal reactions, and then \textbf{LUX} to explore the emotional longing, usefulness, and friction present in their current relationships. \\
    \textbf{Team Facilitator} & Run a post-project debrief using FIND and \textbf{LUX} to explore what worked emotionally and interpersonally, specifically addressing the team's shared longing, perceptions of usefulness, and points of friction. \\
    \textbf{Startup Founder} & Use SPARK to inventory team strengths and brainstorm pivot paths, followed by SAGE to stress-test risk scenarios, informed by insights from \textbf{LUX} regarding the team's intrinsic motivations (longing) and potential pain points (friction). \\
    \textbf{Educator} & Integrate the Gnomy framework into leadership or emotional intelligence curricula to foster reflective decision-making, encouraging students to use \textbf{LUX} to analyze the emotional components of complex problems. \\
    \textbf{Mediator} & Apply CROWS and SAGE with conflicting parties to find shared history, individual concerns, and acceptable compromises, enhanced by using \textbf{LUX} to understand the underlying emotional drivers (longing, usefulness, friction) contributing to each party's perspective. \\
    \hline
\end{tabular}

---

\section{Summary: Why the World Needs Gnomy}
In an era of burnout, isolation, polarization, and rapid change, the Gnomy Framework gives people a structured, imaginative way to reconnect with their inner compass and with each other. It invites emotional honesty and strategic thinking into the same space—bridging feeling and action, pain and power, isolation and connection. It empowers people not just to cope, but to cultivate meaning, collaborate wisely, and lead with heart.

---

\begin{thebibliography}{99}
    \bibitem{linehan1993cognitive} Linehan, M. M. (1993). \textit{Cognitive-Behavioral Treatment of Borderline Personality Disorder}. Guilford Press.
    \bibitem{kabat1990full} Kabat-Zinn, J. (1990). \textit{Full Catastrophe Living: Using the Wisdom of Your Body and Mind to Face Stress, Pain, and Illness}. Delta.
    \bibitem{white1990narrative} White, M., & Epston, D. (1990). \textit{Narrative Means to Therapeutic Ends}. W. W. Norton \& Company.
    \bibitem{schwartz1995internal} Schwartz, R. C. (1995). \textit{Internal Family Systems Therapy}. Guilford Press.
    \bibitem{beck1979cognitive} Beck, A. T., Rush, A. J., Shaw, B. F., & Emery, G. (1979). \textit{Cognitive Therapy of Depression}. Guilford Press.
    \bibitem{fonagy2002affect} Fonagy, P., Gergely, G., Jurist, E. L., & Target, M. (2002). \textit{Affect Regulation, Mentalization, and the Development of the Self}. Other Press.
    \bibitem{bowlby1969attachment} Bowlby, J. (1969). \textit{Attachment and Loss, Vol. 1: Attachment}. Attachment and Loss.
    \bibitem{de1997building} de Shazer, S., & Berg, I. K. (1997). \textit{Building Solutions in Brief Therapy}. W. W. Norton \& Company.
    \bibitem{seligman2011flourish} Seligman, M. E. P. (2011). \textit{Flourish: A Visionary New Understanding of Happiness and Well-being}. Free Press.
    \bibitem{ryan2000self} Ryan, R. M., & Deci, E. L. (2000). Self-determination theory and the facilitation of intrinsic motivation, social development, and well-being. \textit{American Psychologist}, 55(1), 68–78.
    \bibitem{schoemaker1995scenario} Schoemaker, P. J. H. (1995). Scenario planning: A tool for strategic thinking. \textit{Sloan Management Review}, 36(2), 25–40.
    \bibitem{meichenbaum1985stress} Meichenbaum, D. (1985). \textit{Stress Inoculation Training}. Pergamon Press.
    \bibitem{hayes1999acceptance} Hayes, S. C., Strosahl, K. D., & Wilson, K. G. (1999). \textit{Acceptance and Commitment Therapy: An Experiential Approach to Behavior Change}. Guilford Press.
    \bibitem{boyd1986destruction} Boyd, J. R. (1986). \textit{Destruction and Creation}. Unpublished manuscript.
    \bibitem{kolb1984experiential} Kolb, D. A. (1984). \textit{Experiential Learning: Experience as the Source of Learning and Development}. Prentice-Hall.
\end{thebibliography}

\end{document}

